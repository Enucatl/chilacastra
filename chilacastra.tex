\documentclass[italian,a4paper,10pt]{article}
\usepackage{babel,verse,url,geometry,multicol}
\usepackage[usenames]{color}
\usepackage[utf8]{inputenc}
\usepackage[T1]{fontenc}
\usepackage{ae,aecompl}
\frenchspacing

%------------- due ambienti per creare elenchi senza spazi supplementari tra i \item successivi
\newenvironment{packed_item}{
\begin{itemize}
  \setlength{\itemsep}{1pt}
  \setlength{\parskip}{0pt}
  \setlength{\parsep}{0pt}
}{\end{itemize}}
\newenvironment{packed_enum}{
\begin{enumerate}
  \setlength{\itemsep}{1pt}
  \setlength{\parskip}{0pt}
  \setlength{\parsep}{0pt}
}{\end{enumerate}}
\newenvironment{subt}{\vspace{-0.5\baselineskip}\small \centering \em}{\em \normalsize \\}
%
%------------- impostazioni pacchetto geometry
%
%----larghezza del testo
\geometry{textwidth=.86\paperwidth,
%
%----altezza del testo
textheight=55\baselineskip }
%
%------------- ridefinizione simbolo per elenchi puntati: en dash
\renewcommand{\labelitemi}{\textbf{--}}
\newcommand{\crusca}{Vocabolario dell'Accademia della Crusca, 1612}
\title{}
\date{}
\begin{document}
\pagestyle{empty}
%------------------------------------------------
\section*{Una ricerca filologica: \emph{Chi la castra la porcella?}}
\emph{Chi la castra la porcella} è una composizione carnascialesca del 1508,
di Marchetto Cara (1470--1525 ca.). Il testo, dopo una ricerca di anni, è
stato ritrovato in un volume\footnote{\emph{I canti carnascialeschi nelle
fonti musicali del XV e XVI sec.}, Federico Ghisi, 1937, pag. 136--137.} della biblioteca di Storia della Musica
dell'Università di padova. L'interpretazione rimane a tratti alquanto
misteriosa, anche perch\'e nel libro non c'è alcun commento. Si dice solo
che questo è \emph{il canto del castratore di porci}, con l'utile
riferimento bibliografico agli \emph{11 libri di frottole} di Ottaviano
Petrucci da Fossombrone, che non è stato possibile rintracciare. \poemtitle[]{Chi la castra la porcella?}
\begin{subt}
Marchetto Cara
\end{subt}

\begin{multicols}{2}
\begin{verse}[\versewidth]
\poemlines{5}
Chi la castra la porcella?\\
Su, su, za, za, ferri acuti.\\
Per tagliar siam pronti tutti,\\
Che bon mastro ognun s'appella.\\!

Chi la castra la porcella?\\
Conza lavez! Ha! Conza lavez!\\
Chi la castra la porcella?\\!

Nostre bolze e ben fornite\\
Possiam star al paragone.\\
Se di noi bisogno havite\\
Pianterem nostro pongione,\\
Poi cum gran discretione\\
Conzarem vostra padella.\\!

Chi la castra la porcella?\\
Conza lavez! Ha! Conza lavez!\\
Chi la castra la porcella?\\!

Vi daremo un bel coperchio\\
Di lavezi fermo e sodo\\
Se poi rotto haveti el cerchio\\
Conzarenlo cum bon modo\\
Cum inzegno e cum tal chiodo\\
che quadrato a tua cappella.\\!

\columnbreak

Chi la castra la porcella?\\
Conza lavez! Ha! Conza lavez!\\
Chi la castra la porcella?\\!

Horsu, za se gli è dirotto\\
Testi e altri che s'incapa\\
Vi metiem senz'altro motto\\
Quatro ponti in una chiapa\\
Se per tempo poi se schiapa\\
Sarem pronti a puntar quella.\\!

Chi la castra la porcella?\\
Su, su, za, za, ferri acuti\\
Per tagliar siam pronti tutti\\
Che bon mastro ognun s'appella.\\!

Chi la castra la porcella?\\
Conza lavez! Ha! Conzala vez!\\
Chi la castra la porcella?\\!
\end{verse}
\end{multicols}

\section*{Note}
Si sono rivelate particolarmente preziose alcune ricerche su internet, in particolare grazie al dizionario dell'Accademia della Crusca del 1612\footnote{\url{http://www.accademiadellacrusca.it/Vocabolario_1612.shtml}}. Riportiamo qui i risultati più pertinenti:
\begin{description}
\item[Chiappa:]\emph{ s. f. (scherz.)} nel linguaggio dei cacciatori, preda, cattura: \emph{fare una chiappa di selvaggina}.\emph{ s. f. (ant.)} pietra sporgente cui ci si può aggrappare: \emph{montar di chiappa in chiappa} (Dante Inf. XXIV, 33).
\item[Incappare:] 2 \emph{(ant.)} prendere, acchiappare | \emph{v. intr. [aus. essere]} incorrere, imbattersi in cosa o persona molesta. Per \emph{rincontrarsi}, \emph{rintopparsi}.
\item[Bolzone:]4 strumento dotato di un punzone usato per macellare i suini mediante un colpo sulla fronte
\item[Sodo:]Diciamo porre, o mettere in sodo, che vale diliberare, stabilire, fermare. Latin. stabilire, firmare. (\crusca)
\item[Schiappa:] E schiappare un legno. vale farne schegge. Lat. \emph{in ensulas dividere}. E quando vogliamo mostrare uno esser grasso, e di bonissima fatta, diciamo, \emph{Egli è grasso, ch' egli schiappa}, quasi s' apre, e crepa, e non cape nela pelle, modo basso.
\end{description}
Un altro documento riferisce sul carattere scherzoso della composizione:
\begin{quote}
 ci sono pure composizioni di carattere popolaresco,  contenenti anche citazioni popolari. A questo proposito ricordiamo \emph{Amerò, non amerò} [\dots] per non parlare di \emph{Chi la castra la porcella}, una sorta di canto carnascialesco zeppo di doppi sensi e di allusioni nemmeno troppo velate.\footnote{\url{www.ertaitalia.it/File/Settembre 2003/Marchetto Cara.doc}}
\end{quote}
\end{document}
